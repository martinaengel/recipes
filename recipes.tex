% Options for packages loaded elsewhere
\PassOptionsToPackage{unicode}{hyperref}
\PassOptionsToPackage{hyphens}{url}
\PassOptionsToPackage{dvipsnames,svgnames,x11names}{xcolor}
%
\documentclass[
  letterpaper,
  DIV=11,
  numbers=noendperiod]{scrartcl}

\usepackage{amsmath,amssymb}
\usepackage{iftex}
\ifPDFTeX
  \usepackage[T1]{fontenc}
  \usepackage[utf8]{inputenc}
  \usepackage{textcomp} % provide euro and other symbols
\else % if luatex or xetex
  \usepackage{unicode-math}
  \defaultfontfeatures{Scale=MatchLowercase}
  \defaultfontfeatures[\rmfamily]{Ligatures=TeX,Scale=1}
\fi
\usepackage{lmodern}
\ifPDFTeX\else  
    % xetex/luatex font selection
\fi
% Use upquote if available, for straight quotes in verbatim environments
\IfFileExists{upquote.sty}{\usepackage{upquote}}{}
\IfFileExists{microtype.sty}{% use microtype if available
  \usepackage[]{microtype}
  \UseMicrotypeSet[protrusion]{basicmath} % disable protrusion for tt fonts
}{}
\makeatletter
\@ifundefined{KOMAClassName}{% if non-KOMA class
  \IfFileExists{parskip.sty}{%
    \usepackage{parskip}
  }{% else
    \setlength{\parindent}{0pt}
    \setlength{\parskip}{6pt plus 2pt minus 1pt}}
}{% if KOMA class
  \KOMAoptions{parskip=half}}
\makeatother
\usepackage{xcolor}
\setlength{\emergencystretch}{3em} % prevent overfull lines
\setcounter{secnumdepth}{-\maxdimen} % remove section numbering
% Make \paragraph and \subparagraph free-standing
\ifx\paragraph\undefined\else
  \let\oldparagraph\paragraph
  \renewcommand{\paragraph}[1]{\oldparagraph{#1}\mbox{}}
\fi
\ifx\subparagraph\undefined\else
  \let\oldsubparagraph\subparagraph
  \renewcommand{\subparagraph}[1]{\oldsubparagraph{#1}\mbox{}}
\fi


\providecommand{\tightlist}{%
  \setlength{\itemsep}{0pt}\setlength{\parskip}{0pt}}\usepackage{longtable,booktabs,array}
\usepackage{calc} % for calculating minipage widths
% Correct order of tables after \paragraph or \subparagraph
\usepackage{etoolbox}
\makeatletter
\patchcmd\longtable{\par}{\if@noskipsec\mbox{}\fi\par}{}{}
\makeatother
% Allow footnotes in longtable head/foot
\IfFileExists{footnotehyper.sty}{\usepackage{footnotehyper}}{\usepackage{footnote}}
\makesavenoteenv{longtable}
\usepackage{graphicx}
\makeatletter
\def\maxwidth{\ifdim\Gin@nat@width>\linewidth\linewidth\else\Gin@nat@width\fi}
\def\maxheight{\ifdim\Gin@nat@height>\textheight\textheight\else\Gin@nat@height\fi}
\makeatother
% Scale images if necessary, so that they will not overflow the page
% margins by default, and it is still possible to overwrite the defaults
% using explicit options in \includegraphics[width, height, ...]{}
\setkeys{Gin}{width=\maxwidth,height=\maxheight,keepaspectratio}
% Set default figure placement to htbp
\makeatletter
\def\fps@figure{htbp}
\makeatother

\KOMAoption{captions}{tableheading}
\makeatletter
\@ifpackageloaded{caption}{}{\usepackage{caption}}
\AtBeginDocument{%
\ifdefined\contentsname
  \renewcommand*\contentsname{Table of contents}
\else
  \newcommand\contentsname{Table of contents}
\fi
\ifdefined\listfigurename
  \renewcommand*\listfigurename{List of Figures}
\else
  \newcommand\listfigurename{List of Figures}
\fi
\ifdefined\listtablename
  \renewcommand*\listtablename{List of Tables}
\else
  \newcommand\listtablename{List of Tables}
\fi
\ifdefined\figurename
  \renewcommand*\figurename{Figure}
\else
  \newcommand\figurename{Figure}
\fi
\ifdefined\tablename
  \renewcommand*\tablename{Table}
\else
  \newcommand\tablename{Table}
\fi
}
\@ifpackageloaded{float}{}{\usepackage{float}}
\floatstyle{ruled}
\@ifundefined{c@chapter}{\newfloat{codelisting}{h}{lop}}{\newfloat{codelisting}{h}{lop}[chapter]}
\floatname{codelisting}{Listing}
\newcommand*\listoflistings{\listof{codelisting}{List of Listings}}
\makeatother
\makeatletter
\makeatother
\makeatletter
\@ifpackageloaded{caption}{}{\usepackage{caption}}
\@ifpackageloaded{subcaption}{}{\usepackage{subcaption}}
\makeatother
\ifLuaTeX
  \usepackage{selnolig}  % disable illegal ligatures
\fi
\usepackage{bookmark}

\IfFileExists{xurl.sty}{\usepackage{xurl}}{} % add URL line breaks if available
\urlstyle{same} % disable monospaced font for URLs
\hypersetup{
  pdftitle={Mary's Favorite Recipes},
  pdfauthor={Mary},
  colorlinks=true,
  linkcolor={blue},
  filecolor={Maroon},
  citecolor={Blue},
  urlcolor={Blue},
  pdfcreator={LaTeX via pandoc}}

\title{Mary's Favorite Recipes}
\author{Mary}
\date{2024-05-13}

\begin{document}
\maketitle

\section{Welcome!}\label{welcome}

\ldots{} to my culinary treasure trove! Within this recipes folder lies
a tantalizing array of dishes waiting to ignite your passion for cooking
and delight your taste buds. From comforting classics to innovative
creations, each recipe is a journey into flavor, crafted with love and
care. Whether you're a seasoned chef or a kitchen novice, embark on a
culinary adventure with us as we explore the art of cooking together.
Let these recipes inspire you to create memorable meals and moments
shared with loved ones around the table.

Here I am showing you \emph{healthy and mainly vegan} recipes to keep
you satisfied and energized. Have fun exploring!

Love, Mary

\subsection{This is what visitors
said}\label{this-is-what-visitors-said}

\begin{quote}
``This recipes website is a culinary goldmine! Every dish I've made has
turned out phenomenally delicious. It's my go-to resource for meal
inspiration and kitchen adventures.'' (Cam, 22 y/o, UK)
\end{quote}

\begin{quote}
``I've been a loyal follower of this recipes website for years, and it
never disappoints. The recipes are not only mouthwatering but also
incredibly easy to follow, even for someone like me who's not a natural
in the kitchen.'' (Holly, 41 y/o, USA)
\end{quote}

\begin{quote}
``I stumbled upon this recipes website recently, and I'm already hooked!
As someone transitioning to a plant-based diet, I appreciate the wide
variety of vegan recipes available here. Each one is a flavor explosion
that proves vegan food can be both nutritious and indulgent.''
(Elizabeth, 16 y/o, Germany)
\end{quote}

\section{Baking}\label{baking}

\subsection{Vegan Brownie}\label{vegan-brownie}

\emph{Ingridients} + 1 cup all-purpose flour + 1 cup coconut sugar or
granulated sugar + 1/2 cup cocoa powder + 1/2 teaspoon baking powder +
1/2 teaspoon salt + 1/2 cup unsweetened applesauce + 1/4 cup melted
coconut oil or vegetable oil + 1 teaspoon vanilla extract + 1/4 cup
dairy-free milk (such as almond, soy, or oat milk) + 1/2 cup dairy-free
chocolate chips (optional)

\emph{Instructions} 1. In a large mixing bowl, whisk together the flour,
coconut sugar, cocoa powder, baking powder, and salt until well
combined. 1. In a separate bowl, combine the applesauce, melted coconut
oil, vanilla extract, and dairy-free milk. Mix until smooth. 1. Pour the
wet ingredients into the dry ingredients and stir until just combined.
Be careful not to overmix. 1. If using chocolate chips, fold them into
the brownie batter. 1. Pour the batter into the prepared baking pan and
spread it out evenly. 1. Bake in the preheated oven for 25-30 minutes,
or until a toothpick inserted into the center comes out mostly clean
with a few moist crumbs attached. 1. Remove the brownies from the oven
and allow them to cool in the pan for 10-15 minutes before slicing and
serving. 1. Enjoy your delicious vegan brownies! Optional: Serve with a
scoop of dairy-free ice cream or a sprinkle of powdered sugar for an
extra treat.

Have fun and enjoy!

\section{Drinks}\label{drinks}

\section{Cooking}\label{cooking}



\end{document}
